\documentclass[a4paper,graphics,11pt]{article}
\pagenumbering{arabic}

\usepackage[margin=1in]{geometry}
\usepackage[utf8]{inputenc}
\usepackage[T1]{fontenc}
\usepackage{lmodern}
\usepackage[ngerman]{babel}
\usepackage{amsmath, tabu}
\usepackage{amsthm}
\usepackage{amssymb}
\usepackage{complexity}
\usepackage{mathtools}
\usepackage{setspace}
\usepackage{graphicx,color,curves,epsf,float,rotating}
\usepackage{tasks}
\usepackage{enumitem}
\setlength{\parindent}{0em}
\setlength{\parskip}{1em}

\usepackage{tikz}

\definecolor{processblue}{rgb}{0.0, 0.72, 0.92}
\definecolor{lightgray}{rgb}{0.83, 0.83, 0.83}

\usetikzlibrary{arrows,positioning,calc,lindenmayersystems,decorations.pathmorphing,intersections}
\tikzstyle{resource}= [draw,minimum size=16pt,inner sep=0pt,fill=processblue]
\tikzstyle{process} = [draw,minimum size=16pt,inner sep=0pt, rounded corners=3pt, fill=lightgray]
\tikzstyle{allocated} = [->,thick,arrows={-latex}]
\tikzstyle{requested} = [->,thick,arrows={-latex}]

\newcommand{\aufgabe}[1]{\subsection*{Aufgabe #1}}
\newcommand{\up}[2]{\mathrel{\overset{\makebox[0pt]{\mbox{\normalfont\tiny #2}}}{#1}}}

\usepackage{listings}
\usepackage{color}

\definecolor{dkgreen}{rgb}{0,0.6,0}
\definecolor{gray}{rgb}{0.5,0.5,0.5}
\definecolor{mauve}{rgb}{0.58,0,0.82}

\lstset{frame=tb,
    language=Java,
    aboveskip=2mm,
    belowskip=2mm,
    showstringspaces=false,
    columns=flexible,
    basicstyle={\small\ttfamily},
    numbers=left,
    numberstyle=\tiny\color{gray},
    keywordstyle=\color{blue},
    commentstyle=\color{dkgreen},
    stringstyle=\color{mauve},
    breaklines=true,
    breakatwhitespace=true,
    tabsize=4,
    literate={ä}{{\"a}}1 {Ä}{{\"A}}1 {ö}{{\"o}}1 {Ö}{{\"O}}1 {ü}{{\"u}}1 {Ü}{{\"U}}1 {ß}{{\ss}}1
}

\begin{document}
\noindent Gruppe \fbox{\textbf{20}}             \hfill Patrick Arens, 377536\\
\noindent Betriebssysteme und Systemsoftware    \hfill Georg Manthey 376854\\
\strut\hfill Benedikt Gerlach, 376944\\
\begin{center}
	\LARGE{\textbf{Übung 6}}
\end{center}
\begin{center}
\rule[0.1ex]{\textwidth}{1pt}
\end{center}

\aufgabe{6.1}

\aufgabe{6.2}

\textbf{a)}

\begin{tikzpicture}[scale=2, node distance=2cm]

    \node (p2)[process] {$P_2$};
    
    \node (A)[resource] [above of= p2] {$R_A$};   
    \node (p1)[process] [right of=A] {$P_1$};
    \node (D)[resource] [right of=p1] {$R_D$};
    \node (p3)[process] [below of=D] {$P_3$};
    \node (B)[resource] [below of=p1] {$R_B$};
    \node (C)[resource] [below of=p3] {$R_C$};
    

    \draw[allocated] (A) -- (p1);
    \draw[allocated] (B) -- (p2);
    \draw[allocated] (D) -- (p3);
    \draw[allocated] (C) -- (p3);
    \draw[requested] (p1) -- (D);
    \draw[requested] (p2) -- (A);
    \draw[requested] (p3) -- (B);
\end{tikzpicture}

Da ein Circular Wait vorliegt, könnte durchaus ein Deadlock auftreten. Es ist auch keine gemeinsame Nutzung von Betriebsmitteln möglich und wir sind auch davon ausgegangen, dass ein Prozess Betriebsmittel anfordert, während er andere Betriebsmittel benutzt. Nun geht allerdings aus der Aufgabenstellung nicht hervor, ob ein Entzug von Betriebsmitteln möglich ist. Wenn ja, liegt kein Deadlock vor, dann könnte nämlich z.B. $P_2$ das Betriebsmittel $R_B$ entzogen werden und $P_3$ könnte seine Arbeit fortsetzen. Wenn ein Entzug nicht möglich wäre, läge ein Deadlock vor da alle Kriterien dafür erfüllt wären.

\textbf{b)}

\begin{enumerate}[label={\roman*)}]
    \item Diese Aussage stimmt nicht, da die Zeit, die ein Prozess zur Ausführung benötigt, beinahe unabhängig von der Zeit, die es Betriebsmittel benötigt, ist (die Zeit muss nur größer als die längste Zeit die der Prozess ein Betriebsmittel braucht sein). Also kann die benötigte Zeit zum ausführen noch so lang sein, es ist völlig unabhängig davon ob Deadlocks auftreten oder nicht.\\
    Bsp:\\
    \begin{tikzpicture}[scale=2, node distance=1.5cm]
        \node (A) [process] {$A$};
        \node (X) [resource] [left of=A] {$X$};
        \node (B) [process] [below of=X] {$B$};
        \node (Y) [resource] [right of=B] {$Y$};

        \draw[requested] (A) -- (X);
        \draw[allocated] (X) -- (B);
        \draw[requested] (B) -- (Y);
        \draw[allocated] (Y) -- (A);
    \end{tikzpicture}
    Sei dies der Zustand zum Zeitpunkt 3, offensichtlich ist ein Deadlock aufgetreten. Nun nehmen wir an, das beide Prozesse die Betriebsmittel $X$ und $Y$ für je eine Zeiteinheit brauchen, während die Prozesse insgesamt je 4 Zeiteinheiten brauchen. Dann ist offensichtlich:
    $$
        t_{XA}\cdot t_{XB} + t_{YA}\cdot t_{YB} + t_{ZA} \cdot t_{ZB} < t_A \cdot t_B
    $$
    aber es ist trotzdem ein Deadlock aufgetreten.
    
    \item Diese Aussage stimmt auch nicht, da z.B. $t_{XA}=t_{YA}=t_{ZA}=t_A-1$ und das gleiche für den Prozess $B$ gelten kann, dabei würde jeder Prozess sofort alle Betriebsmittel anfordern. Wenn nun einer der beiden Prozesse alle Betriebsmittel bekommt, kann er terminieren während der zweite Prozess auf die Betriebsmittel wartet und nachdem der erste Prozess terminiert hat, diese benutzen und selber terminieren.

    \item Diese Aussage stimmt auch nicht, wobei man hier das gleiche Beispiel wie in der Aussage ii) nutzen kann. Dann wäre nämlich $t_{XA}\cdot t_{XB} + t_{YA}\cdot t_{YB} + t_{ZA} \cdot t_{ZB} > t_A \cdot t_B$, aber es tritt nicht unbedingt ein Deadlock auf.

    \item Diese Aussage stimmt auch nicht, hier kann man nun das Beispiel aus Aussage i) nutzen. Es gilt nach dem Beispiel offensichtlich $t_{XA}+t_{YA}+t_{ZA} < t_A$ und $t_{XB} + t_{YB} + t_{ZB} < t_B$, aber es tritt ein Deadlock auf. 
\end{enumerate}



\end{document}