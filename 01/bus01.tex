\documentclass[a4paper,graphics,11pt]{article}
\pagenumbering{arabic}

\usepackage[margin=1in]{geometry}
\usepackage[utf8]{inputenc}
\usepackage[T1]{fontenc}
\usepackage{lmodern}
\usepackage[ngerman]{babel}
\usepackage{amsmath, tabu}
\usepackage{amsthm}
\usepackage{amssymb}
\usepackage{complexity}
\usepackage{mathtools}
\usepackage{setspace}
\usepackage{graphicx,color,curves,epsf,float,rotating}
\usepackage{tasks}
\setlength{\parindent}{0em}
\setlength{\parskip}{1em}

\newcommand{\aufgabe}[1]{\subsection*{Aufgabe #1}}
\newcommand{\up}[2]{\mathrel{\overset{\makebox[0pt]{\mbox{\normalfont\tiny #2}}}{#1}}}

\usepackage{listings}
\usepackage{color}

\definecolor{dkgreen}{rgb}{0,0.6,0}
\definecolor{gray}{rgb}{0.5,0.5,0.5}
\definecolor{mauve}{rgb}{0.58,0,0.82}

\lstset{frame=tb,
    language=Java,
    aboveskip=2mm,
    belowskip=2mm,
    showstringspaces=false,
    columns=flexible,
    basicstyle={\small\ttfamily},
    numbers=left,
    numberstyle=\tiny\color{gray},
    keywordstyle=\color{blue},
    commentstyle=\color{dkgreen},
    stringstyle=\color{mauve},
    breaklines=true,
    breakatwhitespace=true,
    tabsize=4,
    literate={ä}{{\"a}}1 {Ä}{{\"A}}1 {ö}{{\"o}}1 {Ö}{{\"O}}1 {ü}{{\"u}}1 {Ü}{{\"U}}1 {ß}{{\ss}}1
}

\begin{document}
\noindent Gruppe \fbox{\textbf{20}}             \hfill Patrick Arens, 377536\\
\noindent Betriebssysteme und Systemsoftware    \hfill Georg Manthey 376854\\
\strut\hfill Benedikt Gerlach, 376944\\
\begin{center}
	\LARGE{\textbf{Übung 1}}
\end{center}
\begin{center}
\rule[0.1ex]{\textwidth}{1pt}
\end{center}

\aufgabe{1.1}
\textbf{a)}

\begin{verbatim}
echo "BuS 2016: Abgabe der 1. Uebung am 6.5" | sed -e s/6/7/
\end{verbatim}

\textbf{b)}
\textbf{TODO}

Der Befehl bewirkt folgendes: \textit{-d ’ ’} beschreibt das zeichen das die felder trennt. \textit{-f 1} = nur das erste Feld ausgeben. \textit{d*} = ka

\textbf{c)}

Der Befehl :

\begin{verbatim}
grep -A3 -E ^[[:alpha:]]\+[[:space:]]\+[[:alpha:]]\+$ emails 
| grep -B1 -A1 -E ^[[:alpha:]]\+[[:space:]]\+[[:digit:]]\+$
\end{verbatim}



Erkennt die Adresse:

Arthur Dent\\
Galaxy 7\\
74369 Third Orbit

Welche sich in den Zeilen 10034 bis 19036 befindet. Die E-Mail wurde von realArthurDent@posteo.de verschickt und von emily.saunders@mostlyharmless.com erhalten.\\

\aufgabe{1.2}
\textbf{a)}

\begin{verbatim}
tr -d '"?.!:;,+\&‘' < wotw.txt > wotwNeu.txt
tr -s " " < wotwNeu.txt > wotwNeu2.txt
\end{verbatim}

Zeile 1: Entfernt alle " ? . ! : ; , +  \& ‘  und speichert die Datei unter wotwNeu.txt\\
Zeile 2: Entfernt alle sich wiederholenden Leerzeichen aus wotwNeu.txt und speichert die Datei unter wotwNeu2.txt\\


\textbf{b)}\\

\begin{verbatim}
tr ' ' '\n' < wotw.txt | grep road | wc -l
\end{verbatim}



Das Ergebnis ist 122.

\textbf{c)}\\

\begin{verbatim}
tr -c '[:alnum:]' '[\n*]' < wotw.txt | sort | uniq -c | sort -nr | head  -10
\end{verbatim}


Das Ergebnis ist 12223 absolute Häufigkeit, 4417 the, 2373 and, 2284 of, 1554 a, 1300 I, 1160 to, 924 in, 853 was und 754 that.

\aufgabe{1.3}
\textbf{a)}

Ein Systemcall bzw. Syscall ist eine Methode um vom Betriebssystem bereitgestellte Funktionalitäten auszuführen, wie zum Beispiel das Schreiben einer Datei. Dabei wird die Kontrolle vom Programm an den Kernel übergeben.

\textbf{b)}
\textbf{TODO}

\textit{execve}

\textit{open} = Aufruf einer Datei File Management

\textit{stat}

\textit{mmap}

\textbf{c)}

\textit{strace} = diagnostic tool welchen den Prozess mit dem es gestartet wurde überwacht und alle syscalls die es aufruft speichert.

\textbf{d)}
\textbf{TODO}

strace ls /etc -C -e trace=execve,stat,lstat,fstat,open,openat,getdents,readdir

strace ls -l /etc -C -e trace=execve,stat,lstat,fstat,open,openat,getdents,readdir

(weiterleitung der ausgabe mit > )

\aufgabe{1.4}

\textbf{a)}

\textit{str[1] = ’0’;} $\Longrightarrow$ setzt den zweiten Eintrag des char Arrays str auf 0. str hat nun die werte 102.

\textbf{b)}

\textit{i = *(list + 3);} $\Longrightarrow$ setzt den wert von list[3] auf 7

\textbf{c)}

\textit{pi =  \&list[i];} $\Longrightarrow$ Durch den Befehl wird versucht den Zeiger pi auf den int wert von list[7] zu zeigen. Da list aber nur 4 Elemente hat ist dieser Befehl nicht möglich.

\textbf{d)}

\textit{*pi = 42;} $\Longrightarrow$ Auf den Speicherbereich auf den pi zeigt soll der int wert 42 geschrieben werden. Da pi aber noch nicht initialisiert ist, könnte der Compiler einen Fehler werfen.

\textbf{e)}

\textit{list = pi;} $\Longrightarrow$ Durch diesen Befehl soll die Basisadresse von list auf die Adresse auf die der Zeiger von pi zeigt gelegt werden, da das Array nun aber keinen zusammenhängenden Speicherbereich mehr hat ist das ein Fehler.

\textbf{f)}

\textit{i = *(pi + 2);} $\Longrightarrow$ Da pi immer noch nicht initialisiert ist, ist es unklar worauf pi zeigt. Demnach ist der Zuweisung auf den 2 Speicherbereich nach pi nicht möglich.

\textbf{g)}

\textit{str[3]=’4’;} $\Longrightarrow$ Da str nur 3 Einträge hat ist die abfrage des vierten Eintrags nicht möglich.

\aufgabe{1.5}


\textbf{a)}\textbf{TODO}

Rundungsproblem float zu int.

\textbf{b)}\\

\lstinputlisting{1.5d.txt}

\textbf{c)}

Es ist möglich das Programm ohne explizite Typkonvertierung zu lösen indem man 9/5 durch 1.8 ersetzt.

\aufgabe{1.6}
\textbf{TODO}

\end{document}